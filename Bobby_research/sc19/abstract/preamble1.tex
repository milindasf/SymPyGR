%\usepackage{algorithmic}
\usepackage{algorithm}
\usepackage[noend]{algpseudocode}
\usepackage{array}
\usepackage{amsfonts}
\usepackage{graphicx}
\usepackage{epstopdf}
\usepackage{amssymb}
\usepackage{tikz,pgfplots}
\usetikzlibrary{snakes,arrows,shapes,trees}
\usepackage[position=top,aboveskip=5pt,labelformat=empty]{subfig}
\usepackage{xcolor, graphicx}
\usepackage{amssymb,amsmath,amsthm}
\usepackage{amsopn}
\usepackage{listings}
\usepackage{adjustbox}
\usepackage{longtable}
\usepackage{multirow}
%\usepackage{subcaption}
\DeclareMathOperator{\diag}{diag}

\usepackage{minted}
%\usepackage[outputdir=build]{minted}
\usepackage{amssymb}
\usepackage{pgfplots}
\usepackage{pgfplotstable}
\usetikzlibrary{arrows,shapes,plotmarks}
\pgfplotsset{compat=1.8}

\usepackage{soul}

\usepackage[pdfinfo={Title={Massively Parallel Simulations of Binary Black Hole Intermediate Mass Ratio Inspirals},Subject={SC'18 Applications Paper},Author={Anonymous},pKeywords={Computational Relativity,Einstein Equations,Adaptive Mesh Refinement,Finite Differencing,Wavelet Adaptive Multiresolution}},bookmarks=false,colorlinks=true,urlcolor=darkgray,linkcolor=darkgray,citecolor=violet]{hyperref}

% correct bad hyphenation here
\hyphenation{op-tical net-works semi-conduc-tor}

\definecolor{bgblue}{RGB}{245,243,253}

%\newcommand{\Summit}{\href{https://www.olcf.ornl.gov/summit/}{Summit}}
%\newcommand{\Aurora}{\href{http://aurora.alcf.anl.gov/}{Aurora}}
%\newcommand{\Stampede}{\href{https://www.tacc.utexas.edu/stampede/}{Stampede}}
%\newcommand{\Titan}{\href{https://www.olcf.ornl.gov/titan/}{Titan}}

\newcommand{\mynote}[3]{
    \textcolor{#2}{\fbox{\bfseries\sffamily\scriptsize#1}}
        {\small$\blacktriangleright$\textsf{\emph{#3}}$\blacktriangleleft$}
}


\def\TT{{T}}
\def\SS{{step}}

\newcommand{\dn}[1]{\mynote{David}{magenta}{#1}}
\newcommand{\mf}[1]{\mynote{Milinda}{blue}{#1}}
\newcommand{\hs}[1]{\mynote{Hari}{olive}{#1}}
\newcommand{\ewh}[1]{\mynote{Eric}{red}{#1}}

\newcommand{\dendro}{\textsc{Dendro}}
\newcommand{\NLSM}{\textsc{NLSM}}
\newcommand{\dendrogr}{\textsc{Dendro-GR}}
\newcommand{\HAD}{\textsc{Had}}
\newcommand{\et}{\textsc{ET}}
\newcommand{\bsolver}{\texttt{bssnSolver}}
\newcommand{\BSSN}{BSSNKO}


\newcommand{\oTo}{\textsc{o2o}}
\newcommand{\oTn}{\textsc{o2n}}

\newcommand{\subsubsubsection}[1]{\paragraph{#1}\mbox{}}
\setcounter{secnumdepth}{4}
\setcounter{tocdepth}{4}



\theoremstyle{plain}
\newtheorem{thm}{Theorem}[section]
\newtheorem{lem}[thm]{Lemma}
\newtheorem{prop}[thm]{Proposition}
\newtheorem*{cor}{Corollary}

\theoremstyle{definition}
\newtheorem{defn}{Definition}[section]
\newtheorem{conj}{Conjecture}[section]
\newtheorem{exmp}{Example}[section]

\theoremstyle{remark}
\newtheorem*{rem}{Remark}
%\newtheorem*{note}{Note}

\pgfplotsset{compat=1.13}

\hyphenation{op-tical net-works semi-conduc-tor}

%% user command
\newcommand{\mvec}{\textsc{matvec}}
\newcommand{\tsort}{\textsc{TreeSort}}
\newcommand{\tsearch}{\textsc{TreeSearch}}
\newcommand{\tcons}{\textsc{TreeConstruction}}
\newcommand{\dtcons}{\textsc{DistTreeConstruction}}
\newcommand{\taux}{\textsc{AuxiliaryOctants}}
\newcommand{\tbal}{\textsc{TreeBalancing}}
\newcommand{\tdbal}{\textsc{DistTreeBalancing}}
\newcommand{\tghost}{\textsc{ComputeGhostOctants}}
\newcommand{\teToe}{\textsc{BuildOctantToOctant}}
\newcommand{\teTon}{\textsc{BuildOctantToNodal}}

\newcommand{\hilbertcurve}{\textsc{Hilbert}}
\newcommand{\mortoncurve}{\textsc{Morton}}

\newcommand{\dgn}{\textit{octant local nodes}}
\newcommand{\cgn}{\textit{shared octant nodes}}
\newcommand{\unzip}{\textit{unzip}}
\newcommand{\zip}{\textit{zip}}
\newcommand{\remesh}{\textit{re-mesh}}
\newcommand{\igt}{\textit{inter-grid transfer}}

\newcommand{\tsortmodified}{\textsc{TreeSortModified}}
\newcommand{\dtsortmodified}{\textsc{DistTreeSortModified}}
\newcommand{\tbucket}{\textsc{SFC\_Bucketing}}
\newcommand{\maxDepth}{\textsc{maxdepth}}
%\newcommand{\tsort_cons}{\textsc{OctreeConstruction}}
\newcommand{\ssort}{\textsc{SampleSort}}
\newcommand{\dsort}{\textsc{DistTreeSort}}
\newcommand{\note}[1]{\noindent\emph{\textcolor{purple}{hs: #1}}}
\algrenewcommand\algorithmicrequire{\textbf{Input:}}
\algrenewcommand\algorithmicensure{\textbf{Output:}}
\algrenewcommand\algorithmicforall{\textbf{parallel for}}

\newcommand{\Intel}{Intel\textsuperscript{\textregistered}\xspace}
\newcommand{\snb}{Xeon\texttrademark\xspace}
\newcommand{\xphi}{Xeon Phi\texttrademark\xspace}
\newcommand{\Summit}{\href{https://www.olcf.ornl.gov/summit/}{Summit}}
\newcommand{\Aurora}{\href{http://aurora.alcf.anl.gov/}{Aurora}}
%\newcommand{\Stampede}{\href{https://www.tacc.utexas.edu/stampede/}{Stampede}}
\newcommand{\Stampede}{\href{https://portal.tacc.utexas.edu/user-guides/stampede2}{Stampede2}}
\newcommand{\Titan}{\href{https://www.olcf.ornl.gov/titan/}{Titan}}
\newcommand{\ET}{\href{https://einsteintoolkit.org/}{\textsc{Einstein Toolkit}}}

\newcommand{\norm}[1]{\left\lVert#1\right\rVert}
\newcommand{\octdomain}{\mathcal{N}^3}
%%

\newcommand{\Tau}{\mathcal{T}}

\newdimen\HilbertLastX
\newdimen\HilbertLastY
\newcounter{HilbertOrder}

\def\DrawToNext#1#2{%
  \advance \HilbertLastX by #1
  \advance \HilbertLastY by #2
  \pgfpathlineto{\pgfqpoint{\HilbertLastX}{\HilbertLastY}}
  % Alternative implementation using plot streams:
  % \pgfplotstreampoint{\pgfqpoint{\HilbertLastX}{\HilbertLastY}}
}

% \Hilbert[right_x,right_y,left_x,left_x,up_x,up_y,down_x,down_y]
\def\Hilbert[#1,#2,#3,#4,#5,#6,#7,#8] {
  \ifnum\value{HilbertOrder} > 0%
  \addtocounter{HilbertOrder}{-1}
  \Hilbert[#5,#6,#7,#8,#1,#2,#3,#4]
  \DrawToNext {#1} {#2}
  \Hilbert[#1,#2,#3,#4,#5,#6,#7,#8]
  \DrawToNext {#5} {#6}
  \Hilbert[#1,#2,#3,#4,#5,#6,#7,#8]
  \DrawToNext {#3} {#4}
  \Hilbert[#7,#8,#5,#6,#3,#4,#1,#2]
  \addtocounter{HilbertOrder}{1}
  \fi
}

% \hilbert((x,y),order)
\def\hilbert((#1,#2),#3){%
  \advance \HilbertLastX by #1
  \advance \HilbertLastY by #2
  \pgfpathmoveto{\pgfqpoint{\HilbertLastX}{\HilbertLastY}}
  % Alternative implementation using plot streams:
  % \pgfplothandlerlineto
  % \pgfplotstreamstart
  % \pgfplotstreampoint{\pgfqpoint{\HilbertLastX}{\HilbertLastY}}
  \setcounter{HilbertOrder}{#3}
  \Hilbert[1mm,0mm,-1mm,0mm,0mm,1mm,0mm,-1mm]
  \pgfusepath{stroke}%
}


\ifpdf
  \DeclareGraphicsExtensions{.eps,.pdf,.png,.jpg}
\else
  \DeclareGraphicsExtensions{.eps}
\fi

%strongly recommended
\numberwithin{thm}{section}

\definecolor{cpu3}{HTML}{F44336}
\definecolor{cpu4}{HTML}{2196F3}
\definecolor{cpu1}{HTML}{4CAF50}
\definecolor{cpu2}{HTML}{FFC107}
\definecolor{gpu3}{HTML}{EF9A9A}
\definecolor{gpu4}{HTML}{90CAF9}
\definecolor{gpu1}{HTML}{A5D6A7}
\definecolor{gpu2}{HTML}{FFE082}

\definecolor{cpu5}{HTML}{9932CC}

\definecolor{sq_b1}{RGB}{37,52,148}
\definecolor{sq_b2}{RGB}{44,127,184}
\definecolor{sq_b3}{RGB}{65,182,196}
\definecolor{sq_b4}{RGB}{127,205,187}
\definecolor{sq_b5}{RGB}{199,233,180}
\definecolor{sq_b5}{RGB}{255,255,204}

\definecolor{sq_r1}{RGB}{189,0,38}
\definecolor{sq_r2}{RGB}{240,59,32}
\definecolor{sq_r3}{RGB}{253,141,60}
\definecolor{sq_r4}{RGB}{254,178,76}
\definecolor{sq_r5}{RGB}{254,217,118}
\definecolor{sq_r6}{RGB}{255,255,178}

\definecolor{sq_g1}{RGB}{0,104,55}
\definecolor{sq_g2}{RGB}{49,163,84}
\definecolor{sq_g3}{RGB}{120,198,121}
\definecolor{sq_g4}{RGB}{173,221,142}
\definecolor{sq_g5}{RGB}{217,240,163}
\definecolor{sq_g6}{RGB}{255,255,204}


\definecolor{div_c1}{RGB}{230,171,2}
\definecolor{div_c2}{RGB}{102,166,30}
\definecolor{div_c3}{RGB}{231,41,138}
\definecolor{div_c4}{RGB}{117,112,179}
\definecolor{div_c5}{RGB}{217,95,2}
\definecolor{div_c6}{RGB}{27,158,119}
\definecolor{div_c7}{RGB}{215,48,39}





\definecolor{lineclr}{RGB}{0,0,0}
\definecolor{utorange}{RGB}{0,0,255}
\definecolor{utsecblue}{RGB}{255,255,0}
\definecolor{utsecgreen}{RGB}{255,0,0}
\definecolor{red!15}{RGB}{0,255,255}
\definecolor{fillclr5}{RGB}{0,255,0}
\definecolor{fillclr6}{RGB}{255,0,255}
\definecolor{fillclr7}{RGB}{255,255,255}
\definecolor{fillclr8}{RGB}{0,0,0}





\def\drawcubeI(#1,#2,#3,#4,#5){ % x,y,z,sz,line color
\coordinate (O) at (#1,#2,#3);
\coordinate (A) at (#1,#2+#4,#3);
\coordinate (B) at (#1,#2+#4,#3+#4);
\coordinate (C) at (#1,#2,#3+#4);
\coordinate (D) at (#1+#4,#2,#3);
\coordinate (E) at (#1+#4,#2+#4,#3);
\coordinate (F) at (#1+#4,#2+#4,#3+#4);
\coordinate (G) at (#1+#4,#2,#3+#4);
\draw[#5] (O) -- (C) -- (G) -- (D) -- cycle;% Bottom Face
\draw[#5] (O) -- (A) -- (E) -- (D) -- cycle;% Back Face
\draw[#5] (O) -- (A) -- (B) -- (C) -- cycle;% Left Face
\draw[#5] (D) -- (E) -- (F) -- (G) -- cycle;% Right Face
\draw[#5] (C) -- (B) -- (F) -- (G) -- cycle;% Front Face
\draw[#5] (A) -- (B) -- (F) -- (E) -- cycle;% Top Face
}


\def\drawcubeII(#1,#2,#3,#4,#5,#6,#7){ % x,y,z,sz,line color,fill color,opacity
\coordinate (O) at (#1,#2,#3);
\coordinate (A) at (#1,#2+#4,#3);
\coordinate (B) at (#1,#2+#4,#3+#4);
\coordinate (C) at (#1,#2,#3+#4);
\coordinate (D) at (#1+#4,#2,#3);
\coordinate (E) at (#1+#4,#2+#4,#3);
\coordinate (F) at (#1+#4,#2+#4,#3+#4);
\coordinate (G) at (#1+#4,#2,#3+#4);
\draw[#5,fill=#6,opacity=#7] (O) -- (C) -- (G) -- (D) -- cycle;% Bottom Face
\draw[#5,fill=#6,opacity=#7] (O) -- (A) -- (E) -- (D) -- cycle;% Back Face
\draw[#5,fill=#6,opacity=#7] (O) -- (A) -- (B) -- (C) -- cycle;% Left Face
\draw[#5,fill=#6,opacity=#7] (D) -- (E) -- (F) -- (G) -- cycle;% Right Face
\draw[#5,fill=#6,opacity=#7] (C) -- (B) -- (F) -- (G) -- cycle;% Front Face
\draw[#5,fill=#6,opacity=#7] (A) -- (B) -- (F) -- (E) -- cycle;% Top Face
}




\def\drawNodes(#1,#2,#3,#4,#5,#6,#7){ % x_min,x_max,y_min,y_max,z_min,z_max,min+stepSz
\foreach \x in {#1,#7,...,#2}{
	\foreach \y in {#3,#7,...,#4}{
		\foreach \z in {#5,#7,...,#6}{
				\draw[fill=red!60] (\x,\y,\z) circle (0.15);
				}
			}
	}				
		
}


\makeatletter
\newcommand\resetstackedplots{
\makeatletter
\pgfplots@stacked@isfirstplottrue
\makeatother
\addplot [forget plot,draw=none] coordinates{(16,0) (32,0) (64,0) (128,0) (256,0) (512,0) (1024,0) (2048,0) (4096,0) (8192,0) (16384,0) (32768,0) (65536,0) (131072,0) (262144,0)};
}
\makeatother

\makeatletter
\newcommand\resetstackedplotsOne{
\makeatletter
\pgfplots@stacked@isfirstplottrue
\makeatother
\addplot [forget plot,draw=none] coordinates{(16,0) (32,0) (64,0) (128,0) (256,0) (512,0) (1024,0) (2048,0) (4096,0)};
}
\makeatother

\makeatletter
\newcommand\resetstackedplotsTwo{
\makeatletter
\pgfplots@stacked@isfirstplottrue
\makeatother
\addplot [forget plot,draw=none] coordinates{(32,0) (64,0) (128,0) (256,0) (512,0) (1024,0) (2048,0) (4096,0) (8192,0) (16384,0) (32768,0) (65536,0) };
}
\makeatother


\makeatletter
\newcommand\resetstackedplotsThree{
\makeatletter
\pgfplots@stacked@isfirstplottrue
\makeatother
\addplot [forget plot,draw=none] coordinates{(2,0) (4,0) (8,0) (16,0) (32,0) (64,0)};
}
\makeatother


\makeatletter
\newcommand\resetstackedplotsFour{
\makeatletter
\pgfplots@stacked@isfirstplottrue
\makeatother
\addplot [forget plot,draw=none] coordinates{(4,0) (8,0) (16,0) (32,0) (64,0)};
}
\makeatother
